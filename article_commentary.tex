\documentclass{beamer}
\usetheme{Madrid}
\usepackage{makecell}

\usepackage{graphicx} % Required for inserting images

\title{Deep Learning}
\subtitle{PEC5 - Estado del Arte}
\author{Judith Urbina Córdoba}
\institute{UOC}
\date{\today}

\begin{document}

\begin{frame}
    \titlepage
\end{frame}

\section{Datos básicos del artículo}
\subsection{Referencia completa}
\begin{frame}
    \frametitle{Título y detalles técnicos del artículo}
    Título: 3D Gaussian Splatting for Real-Time Radiance Field Rendering
    Autores: \begin{itemize}
        \item  Bernhard Kerbl (Université Côté d'Azur)
        \item Georgios Kopanas (Université Côté d'Azur)
        \item Thomas Leimkühler (Max-Planck-Institut für Informatik)
        \item George Drettakis (Université Côté d'Azur)
    \end{itemize}
    Publicación: 8 Agosto de 2023 en ACM (Association for Computing Machinery), Volumen 42, Número 4, Artículo 1
\end{frame}

\subsection{Descripción de la temática y novedades}
\begin{frame}
    \frametitle{Descripción de la temática y novedades del artículo}
    Temática: Aplicación del Machine Learning en la Reconstrucción 3D de escenarios mediante NeRF (neural Radiance Field)
    Novedades:
   \begin{enumerate}[I]
        \item Representación con variables Gaussianas 3D diferenciables
        \item Como input toman puntos SfM
       \item Usan coeficientes de armónicos esféricos (SH coefficients)
       \item Se han basado en principios de real-time-rendering
   \end{enumerate}
\end{frame}

\section{Parte experimental del artículo}
\subsection{Resumen de la parte experimental}

\begin{frame}
\frametitle{Implementación del método}
La programación se hace con Python (Pytorch), CUDA, SIBR.
Esquema:
\begin{table}
    \begin{tabular}{l| c}
        Datasets usados &  \makecell{Sintéticos con \textit{Blender}, \\ TanksAndTemples,  \\Conjuntos de Mip-NeRF360}\\
        \hline
        Algoritmos comparados & \makecell{Ground Truth, 3D Gaussian Splitting 7-K, \\3D Gaussian Splatting-30k, Mip-NeRF360, \\InstantNGP, Plenoxels}\\
        \hline
        Métricas usadas & \makecell{SSIM, PSNR, LPIPS, Tiempo de entrenamiento,\\FPS y Memoria}\\
        \hline
        Estudios de Ablación & \makecell{BW Limitado, Random Init, No-Split,\\No-SH, No-Clone, Isotrópico, Entero}
    \end{tabular}
    \caption{Parte experimental del estudio}
\end{table}
    
\end{frame}


\section{Conclusiones y resumen crítico del artículo}
\subsection{Conclusiones y resumen crítico}
\begin{frame}
\frametitle{Conclusiones y resumen crítico del artículo}
    \begin{columns}
    \column{0.5\textwidth}
    \begin{itemize}
        \item Velocidad de renderización equiparable a los mejores métodos previos
        \item Calidad equiparable a los mejores métodos previos
    \end{itemize}
    \column{0.5\textwidth}
    \begin{itemize}
        \item  En las regiones con mala visualización hay fallas
        \item Requiere de más memoria que otros métodos
    \end{itemize}
 
    \end{columns}
Mezcla lo mejor de los métodos encontrados hasta el momento.\\
Plantean hacer mesh reconstructions en el futuro con su método.
\end{frame}


\end{document}
